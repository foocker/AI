\chapter{机器学习}
\label{chap:mlm}
本章包含了机器学习的经典算法.
经典的机器学习算法有很多库都已经实现,我们没必要所有都去造轮子,我的选择是理解其数学部分,使用现有的库sklearn,并在实践中分析理论与和实际的差距。假设我们对理论和库的调用都不太熟悉,实际上sklearn的document本身就是一个很好的学习地方,那里包含了算法的相关参考文献。后面的章节我们首先以这种方式来学习经典机器学习。

开始页面:\href{http://scikit-learn.org/stable/user_guide.html#user-guide}{sklearn-user-guide}

\begin{description}
    \item[1]监督学习
        \begin{description}
            \item[1.1] 一般线性模型
            \item[1.3] 支持向量机
            \item[1.5] 随机梯度下降法
            \item[1.7] 高斯过程
            \item[1.11] 集成方法
            \item[1.12] 多类和多标签
            \item[1.13] 特征选择
            \item[1.17] 神经网络(监督)
        \end{description} 
    \item[2]非监督学习
        \begin{description}
            \item[2.1] 高斯混合模型
            \item[2.2] 流形学习
            \item[2.3] 聚类
            \item[2.9] 神经网络模型(非监督)
        \end{description}
    \item[3]模型选择和评估  
    \item[4]数据处理
        \begin{description}
            \item[4.1] Pipeline and FeatureUnion:组合估计
            \item[4.2] 特征提取
            \item[4.3] 数据预处理
            \item[4.4] 非监督降维
            \item[4.5] 随机投影
            \item[4.6] 核近似
            \item[4.7]  Pairwise metrics, Affinities and Kernels
            \item[4.8] 变换目标值
        \end{description}
    \item[5]数据导入
    \item[6]大数据 
\end{description}

以上是sklearn指导文档首页的部分目录,现在我们随机选择一些东西学习,比如我这里选择了接下来的四节内容。这只是一个初步的学习,剩下的就是在实践中不断的深化理解实际和理论上的差别,然后再反过来思考理论上的问题。这部分的理论相对简单,但这些优化方法却是人工智能的基础。

\section*{Regression}
\label{sec:lr}
进入一般线性模型,玲琅满目,眼花缭乱。
\subsection{Logistic regression}
逻辑回归是一个二分类概率模型,其很容易扩展到多元情形,它将特征向量映射为一个概率向量,其每个分量表示特征属于其对应标签的概率。模型可表示如下:
\begin{equation}
    P^{LR}(W)=C\sum_{i=1}^{n}log(1+e^{-y_{i}W^{T}\mathrm{x_i}}) + 1/2W^{T}W
\end{equation}

其中$\{\mathrm{x_i},y_i\}_{i=1}^{n}$表示数据以及其标签,$\mathrm{x_{i}}\in R^{m}, y_i \in \{1, -1\}.C>0,W$是要学习的参数。

给定数据及其标签,我们可以用如下公式来表示条件概率。
\begin{equation}
    P_{W}(y=\pm 1|\mathrm{x})=\frac{1}{1+e^{-yW^{T}\mathrm{x}}}
\end{equation}

根据极大似然原理,我们很容易由(2.2)得到(2.1),如果我们去掉(2.1)的$1/2W^{T}W$,这个多余的东西其实就是正则项,用来限制参数$W$,防止过拟合的技巧。这个后面详说。在实际情况中,往往需要很多额外起脚来使模型更加实用。

现在的问题是如何得到模型参数$W,C$?

答案:Coordinate descent approch, quasi-Newton method,iterative scaling method, exponetiatel gradient ...如果你学过数值分析的话,你会觉得很多似曾相识,如果你学过凸分析的话,你会觉得很亲切,随着学习的深入,我们会逐渐建立更清晰的理论框架。现在不妨将视角转向SVM。

\href{https://en.wikipedia.org/wiki/Linear_regression}{Linear regression}
\href{https://en.wikipedia.org/wiki/Logistic_regression}{Logisticregression}

\section*{支持向量机}
\label{sec:SVM}
SVM,一个二分类线性模型,简单的说,就是我们高中遇见的线性规划问题的推广。我们知道直线$y=kx+b$将平面$xy$分成两部分,其实也就是两类,一类在“上面”,一类在“下面”。现在我们的情况只是在维度上增加了,也就是寻找一个超平面$y=Wx+b$能将数据分类出来。模型可表示如下:

\begin{equation}
    P^{SVM}(W)=C\sum_{i=1}^{n}max(1-y_iW^{T}\mathrm{x_i},0)+1/2W^{T}W
\end{equation}

很明显超平面是依赖训练数据的。从文档上看,该分类其的好处有:

    \begin{description}
        \item[.] 高维空间上比较高效。
        \item[.] 对特征维度大于样本量时,仍然有效(大过多时,需要适当的选择核函数和正则项)。
        \item[.] 用部分数据来得到决策函数,也即支持向量,能减少内存。
    \end{description}

上面只是最简单的情形,多分类,多元回归呢?

我们先来看看文档里的情况,
对于多分类,从文档里我们了解到两种方法:SVC的“one-against-one”,$n$类标签构建
$\binom{n}{2}$个分类器;LinearSVC的“one-vs-the-rest”,训练$n$个模型。

对于回归问题,其对应的模块名叫SVR。自行查看即可。现在的问题是:怎么从分类模型过度到回归模型呢?完整想出来,似乎还是有点难度,但是我们看到网页所给参看文献\href{http://citeseerx.ist.psu.edu/viewdoc/summary?doi=10.1.1.114.4288}{SVR},好了,又到了真正学习的时候了。

\textcolor{red}{此段讲理论,以及代码分析。}

然而实际上我们对多分类问题的处理方式还是失望的,我们并不想重复二分类模型,针对这个问题,表明我们该看看1.12节\href{http://scikit-learn.org/stable/modules/multiclass.html}{Multiclass-Multilabel}了。

\textcolor{red}{Maximum Entropy}

蹦,问题又来了,所谓超平面,直观上看,毕竟是个“平”的。实际问题中,数据往往需要用一个弯曲的面才能将其较好的分类出来,这时怎么办呢?自然的我们有两种想法,一种直接把超曲面算出来,但这样不太好,考虑到曲面的表示方式,能控制的范围太小了,而实际变化范围太大;另一种方法就是保持超平面不变,直接映射输入数据,使其能被平面分割。这就是所谓的核技巧。

\textcolor{red}{核技巧讲完了。}

现在我们来看看以上模型该如何学习参数。

随机选择一个Reference:\href{http://www.csie.ntu.edu.tw/~cjlin/papers/maxent_dual.pdf}{Dual coordinate descent for LR and ME}

\section*{Adaboost}
\label{sec:Adaboost}

\section*{高斯混合模型}

\endinput