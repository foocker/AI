% 00_preface.tex
\chapter{前言}
\label{chap:preface}
本文档是我的一个学习过程,学习时间开始于2016年末,打算将其写出来于2017年中。本着不是科班出生,很多方面都不太扎实,包括程序设计,对算法的性能分析,以及囿于匮乏的经验对市场的状况认识还不太合格。将其写出来一来梳理自己,二来也可以利用自己的数学优势以及先行几步减少部分初学者学习上的困惑,最后也能知道自己的不足之处以及理解有误的地方。


首先以词条人工智能开始维基百科之旅:
\href{https://zh.wikipedia.org/wiki/%E4%BA%BA%E5%B7%A5%E6%99%BA%E8%83%BD}{人工智能}。

AI的核心问题包括推理,知识,规划,学习,交流,感知,移动和操作物体等。目前比较流行的方法包括统计方法,计算智能和传统意义的AI。大量应用的人工智能包括搜索和数学优化,逻辑推演。

接着自由选择进入机器学习页面:
机器学习是人工智能的一个分支。在30多年的发展中,已成为一门多领域的交叉学科,涉及概率论,统计学,逼近论,凸分析,计算复杂性理论等多门学科。其算法主要实现从数据中自动分析获得规律,并利用规律对未知数据进行预测。主要分为四类:监督学习(回归分析和统计分类),无监督学习(聚类),半监督学习和增强学习(基于环境行动,以取得最大化的预期利益)。

具体的机器学习算法:
\begin{enumerate}
    \item 构造间隔理论分布:聚类分析和模式识别
    \begin{enumerate}
        \item 人工神经网络
        \item 决策树
        \item 感知器
        \item 支持向量机
        \item 集成学习AdaBoost
        \item 降维与度量学习
        \item 聚类
        \item 贝叶斯分类器
    \end{enumerate}
    \item 构造条件概率:回归分析和统计分类
    \begin{enumerate}
        \item 高斯过程回归
        \item 线性判别分析
        \item 最近邻法
        \item 径向基函数核
    \end{enumerate}
    \item 通过再生模型构造概率密度函数:
    \begin{enumerate}
        \item 最大期望算法
        \item 概率图模型:贝叶斯网和Markov随机场
        \item Generative Topographic Mapping
    \end{enumerate}
    \item 近似推断技术:
    \begin{enumerate}
        \item 马尔科夫链
        \item 蒙特卡洛方法
        \item 变分法
    \end{enumerate}
    \item 最优化:大多数以上方法,直接或间接使用最优化算法。
    
\end{enumerate}

以上是中文版的粗略分类,关于更详尽的分类可参考:\href{https://en.wikipedia.org/wiki/Outline_of_machine_learning}{Outline\_of\_ml}.假设我们已经浏览了此页面上所有内容,包括其链接内容。
这样我们对人工智能整体有了个一个简单的认识:算法庞杂,理论繁多,近几年活跃度很大。一方面方法可以很简单,比如KNN,PCA,BP...,另一方面也可以很复杂,比如变分法,最优传输等。以现有的目光来看,很多基本的问题还需要学者们去解答。包括一个统一的理论框架,对深度黑箱的解释等。另外如此多的算法,在面对实际问题时,往往局限于模型的理想化,以及问题的类型,需要根据实际选择并改装。这也就促使我们选择自己感兴趣的分支,并掌握所需算法的精髓。似乎一下就变成了生命能承受之重了,...似轻非轻...,道路很长。

% \noindent
% 选择:CV(可变NLP)\\
% 工具:TensorFlow,Pytorch,sklearn,numpy...\\
% 理论:相似与度量学习,凸优化,概率统计,最优传输\\
% 问题:怎样学到最少的东西,解决更多的问题?
\indent
\begin{enumerate}
    \item[*] 选择:CV
    \item[*] 工具:TensorFlow,Pytorch,sklearn,Numpy,Opencv...
    \item[*] 理论:相似与度量学习,凸优化,概率统计,最优传输
    \item[*] 问题:怎样学到最少的东西,解决更多的问题?
    % \item[*] 问题2:怎样创造
\end{enumerate}

关于工具的问题,纯粹学习工具本身,是一个挺无聊的过程。带着问题或者项目学习,在一定程度上能减少一些莫名的痛苦。
学习工具之前要有一定的理论基础,单单学习框架是无意义的,假设不从长远来看的话,那就没问题了。理论学习若能结合具体问题进行比较,在面向工程时,感觉会过渡得更加自然,反过来,也可能会给理论研究注入新鲜东西。


\endinput